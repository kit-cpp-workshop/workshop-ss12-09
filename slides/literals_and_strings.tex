\section{Literale}
\begin{frame}{Literale}
    \begin{block}{Was ist ein Literal?}
    Literale sind Bestandteil der Syntax der meisten Sprachen, die dazu dienen, Daten direkt in den Quellcode zu schreiben.
    \end{block}
    \pause
    \begin{block}{Beispiele für Literale in C++}
    \begin{table}
    \center
    \begin{tabular}{ll}
        \toprule
        Typ & Beispiel \\
        \midrule
        char & 'A' \\
        char* (char-Array) & "Hello World!" \\
        int & 42 \\
        float & 1.6e-19 \\
        \bottomrule
    \end{tabular}
    \caption{Literale in C++}
    \end{table}
    \end{block}
\end{frame}

\begin{frame}{Escape-Sequenzen}
    \begin{block}{Escape-Sequenzen in String-Literalen in C++}
    \begin{table}
    \center
    \begin{tabular}{ll}
        \toprule
        Sequenz & Wirkung \\
        \midrule
        \textbackslash n & Neue Zeile \\
        \textbackslash r & Tabulator \\
        \textbackslash\textbackslash & Backslash (\textbackslash) \\
        \textbackslash" & Doppeltes Anführungszeichen (") \\
        \textbackslash' & Einfaches Anführungszeichen (') \\
        \bottomrule
    \end{tabular}
    \end{table}
    \end{block}
\end{frame}

\begin{frame}{Weniger häufig genutzte Literale}
    \begin{block}{Weitere Beispiele für Literale}
        \begin{table}
        \center
        \begin{tabular}{ll}
            \toprule
            Typ & Beispiel \\
            \midrule
            Oktalzahl (int) & 042 \\
            Hexadezimalzahl (int) & 0x732 \\
            \midrule
            \pause
            unsigned int & 42u \\
            unsigned int, Oktal & 042u \\
            \midrule
            \pause
            C++11: Benutzerdefiniert & 42.7\_meter \\
            \bottomrule
        \end{tabular}
        \caption{Weitere Literale in C++}
        \end{table}
    \end{block}
\end{frame}

\section{Strings}