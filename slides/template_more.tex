\section{Etwas mehr zu templates}


\subsection{template argument deduction}

\begin{frame}[t]{template argument deduction}
	Bei einem \enquote{Aufruf eines function templates} kann der Compiler (manche) template-Argumente anhand der Typen der function-call-Argumente bestimmen:
	
	\onslide*<+> { \lstinputlisting[linerange=begin-deduction]{cpp-code/type-deduction.cpp} }
	\onslide<+-> { \lstinputlisting[linerange=begin-]{cpp-code/type-deduction.cpp} }
\end{frame}


\subsection{non-type template parameters}

\begin{frame}[t]{non-type template parameters}
	Bisher: Typen als Parameter der Vorlage\\
	\onslide<2> { Jetzt neu: z.B. \texttt{int} als Parameter der Vorlage }
	
	\onslide*<1> { \lstinputlisting[linerange=int-instantiation]{cpp-code/non-type-template-parameters.cpp} }
	\onslide<2> { \lstinputlisting[linerange=int-]{cpp-code/non-type-template-parameters.cpp} }
\end{frame}

\begin{frame}{non-type template parameters: constraints}
	Der Wert muss eine constant-expression sein, also dem Compiler bekannt!	\\
	Denn sonst kann dieser das Template nicht Instantiieren.
	
	\vspace{2em}
	
	Erlaubte Typen:
	\begin{itemize}
		\item integral
		\item enumeration
		\item exotisch: Referenz oder Poiniter zu »Ding« \emph{mit external linkage}
		\item kein String!
	\end{itemize}
\end{frame}
